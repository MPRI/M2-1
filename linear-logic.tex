\documentclass[a4paper, 11pt]{article}

% Compiler avec XeLaTeX !!

\usepackage{amsmath}
\usepackage{amssymb}
\usepackage{stmaryrd}

\usepackage{unicode-math}
\usepackage{polyglossia}
\setmainlanguage{french}
\usepackage{csquotes} % guillemets

% Changer choix de police ? Palatino est plus assez hipster pour moi
\setmainfont[Mapping=tex-text]{TeX Gyre Pagella}
\setmathfont{Asana-Math.otf}

\usepackage{fullpage} % pour faire tenir les règles sur la page


%% Configuration propre à la logique linéaire

% Arbres de preuves de séquents
\usepackage{bussproofs}

% Opérateurs propres à la logique linéaire
% Mieux que le package cmll car ça utilise la bonne police avec unicode
% tenseur = \otimes, ou additif = \oplus
\newcommand{\avec}{\mathbin{\&}}
\newcommand{\parr}{\mathbin{⅋}}
\newcommand{\ofcourse}{\mathord{!}}
\newcommand{\whynot}{\mathord{?}}

\begin{document}

\title{Logique linéaire et paradigmes logiques du calcul}
\author{Cours MPRI 2.1}
\date{2013--2014}

\maketitle

% Pour rendre les tableaux de règles avec tabular plus agréables.
\renewcommand{\arraystretch}{2}



BROUILLON TRES PRELIMINAIRE.

Synopsis du cours pompé sur la page web :

\enquote{Une analyse fine des calculs de séquents classiques et intuitionnistes permet de concevoir des logiques plus adaptées aux problèmes de l'informatique et de dévélopper, grâce à la correspondance de Curry-Howard des formalismes intermédiaires entre le $\lambda$-calcul et les vrais langages de programmation.

Ce cours a pour but de donner une vision d'ensemble des motivations et des applications d'une de ces logiques, la Logique Linéaire, qui permet une analyse plus fine des processus de démonstration et de calcul, et d'introduire les notions de base des calculs intermédiaires les plus connus. On montrera dans le cours comment ces deux approches se rejoignent, à travers des interprétations calculatoires adaptées.

Ce cours dédie une attention toute particulière aux aspects syntaxiques et calculatoires des formalismes logiques.}


Pour la sémantique, voir le cours \enquote{Modèles de la programmation}, notamment la partie de P.A Melliès.

Intervenants en 2013--2014 : Roberto di Cosmo, Delia Kesner\ldots ajouter calendrier ?


Références :
\begin{itemize}
\item The Linear Logic Primer (Vincent Danos et Roberto Di Cosmo)
\item Proof and Types (Jean-Yves Girard, traduit par Yves Lafont et Paul Taylor)
\item more to come\ldots
\end{itemize}

Ajouter une liste d'articles ?

Les titres des sections ne sont pas fixés. Les suggestions sont bienvenues.

Dans la hiérarchie de la table des matières, on a
\begin{itemize}
\item Section $\simeq$ intervenant (ou thème)
\item Sous-section $\simeq$ Séance de cours
\item Sous-sous-section $\simeq$ partie au sein d'une séance
\end{itemize}

\newpage

\tableofcontents

\newpage

\section{Introduction à la logique linéaire}

\subsection{Du calcul des séquents à la logique linéaire}

L'histoire du programme de Hilbert et de son échec (théorème d'incomplétude) est bien connue. Vers la même période où Gödel montre l'impossibilité d'une preuve de cohérence de $PA$ (l'arithmétique de Peano) dans lui-même, Gentzen parvient à montrer que $PA$ est cohérente par une induction jusqu'à l'ordinal $\varepsilon_0$. Pour cela, il introduit des systèmes de preuves, la \emph{déduction naturelle} et le \emph{calcul des séquents}, et ramène l'absence de contradiction du système à la terminaison d'une procédure d'\emph{élimination des coupures} sur les preuves, s'approchant ainsi du rêve finitiste hilbertien. Contrairement aux systèmes dits \enquote{à la Hilbert}, ceux de Gentzen contiennent peu d'axiomes, mais beaucoup de règles de déduction.

Les sytèmes de preuves que nous allons étudier sont constitués de règles formelles permettant de \emph{dériver} des \emph{jugements}. Un jugement est de la forme $\Gamma \vdash \Delta$, où $\Gamma$ et $\Delta$ sont des \emph{séquents}, c'est-à-dire des listes de formules de la forme $A_1, \ldots, A_n$. (Le symbole $\vdash$ s'appelle \enquote{turnstile} et est prononcé \enquote{thèse} en français.) $\Gamma \vdash \Delta$ signifie informellement \enquote{à partir des hypothèses $\Gamma$, on déduit les conclusions $\Delta$}, sachant que l'interprétation de la virgule et du turnstile sont à préciser en fonction du système.

Les règles du calcul sont écrites sous la forme
\begin{prooftree}
\AxiomC{$A_1$}
\AxiomC{$A_2$}
\AxiomC{\ldots}
\AxiomC{$A_n$}
\RightLabel{\mbox{(nom de la règle)}}
\QuaternaryInfC{$B$}
\end{prooftree}
où les $A_i$ et $B$ sont des jugements, la barre verticale signifiant \enquote{si les jugements $A_i$ sont valides, alors $B$ l'est}. En construisant un arbre dont les noeuds sont des applications de ces règles, on obtient une \emph{preuve} dans le système (aussi appelée dérivation).

Commençons par le calcul des séquents qui se décline en variantes classique (LK) et intuitionniste (LJ).

\subsubsection{Logique classique : le système LK de Gentzen}

En LK, $A_1, \ldots, A_n \vdash B_1, \ldots, B_m$ signifie \enquote{$A_1$ et $A_2$ \ldots et $A_n$ prouve $B_1$ ou $B_2$ \ldots ou $B_m$}, ou encore \enquote{si tous les $A_i$ sont vrais, l'un des $B_j$ sera vrai}. La virgule n'a pas la même signification des 2 côtés du turnstile !

Voici les règles du système LK :

\paragraph{Axiome} L'axiome est la règle permettant d'introduire une prémisse présente dans la liste des hypothèse. C'est la seule règle sans \enquote{hypothèse}\footnote{il faudrait trouver un meilleur nom pour distinguer avec les hypothèses d'un jugement} : les feuilles d'un arbres de preuve sont obligatoirement des axiomes. \\
Elle s'écrit
\begin{prooftree}
\AxiomC{}
\RightLabel{(Ax)}
\UnaryInfC{$\Gamma, A \vdash A, \Delta$}
\end{prooftree}
Interprétation : \ldots

\paragraph{Coupure} Cette règle est en même temps essentielle et inutile (on verra pourquoi après). Elle correspond à l'emploi, dans la preuve d'un théorème, d'un lemme préalablement prouvé.
Elle s'écrit :
\begin{prooftree}
\AxiomC{$\Gamma \vdash A, \Delta$}
\AxiomC{$\Gamma', A \vdash \Delta'$}
\RightLabel{(Cut)}
\BinaryInfC{$\Gamma, \Gamma' \vdash \Delta, \Delta'$}
\end{prooftree}
blabla pour préciser l'interprétation

\paragraph{Règles structurelles} La flemme d'écrire ça\ldots échange + contraction + affaiblissement avec versions à gauche et à droite. Dans mes notes de cours, j'ai :
\begin{itemize}
\item contraction : l'immuabilité de la vérité permet d'utiliser des hypothèses plusieurs fois
\item affaiblissement : élargir les hypothèses ; (permet de faire des manips sur le système lui-même ??)
\end{itemize}
Ces règles peuvent paraître anodines, elles semblent tomber sous le sens. Pourtant, elles vont se révéler très importante.

\paragraph{Règles logiques} Règles permettant l'introduction des connecteurs, manifestant ainsi dans la syntaxe des formules la structure du système logique. Pour avoir un équivalent des règles d'élimination de la déduction naturelle, utiliser intro + coupure.
Copier depuis en bas.
2 présentations, mult et add. équivalence : emploie les règles structurelles.

\begin{prooftree}
\AxiomC{$\Gamma, A \vdash \Delta$}
\UnaryInfC{$\Gamma \vdash \neg A, \Delta$}  
\end{prooftree}
\begin{prooftree}
\AxiomC{$\Gamma \vdash A, \Delta$}
\UnaryInfC{$\Gamma, \neg A \vdash \Delta$}  
\end{prooftree}

\paragraph{Quantification} Pour faire de la logique du 1er ordre.

\paragraph{} Comme exemple, voici une preuve du tiers exclu sans hypothèse en LK :
\begin{prooftree}
  \AxiomC{}
  \RightLabel{(Ax)}
  \UnaryInfC{$A \vdash A$}
  \UnaryInfC{$\vdash A, \neg A$}
  \RightLabel{($\lor_R$)}
  \UnaryInfC{$\vdash A \lor \neg A$}
\end{prooftree}

\paragraph{Jolies propriétés du système LK} En voici quelques-unes simples :
\begin{itemize}
\item Il y a une symétrie dans le calcul entre les 2 côtés du $\vdash$, en passant entre les 2 par la négation. La symétrie se traduit aussi par la dualité entre les connecteurs $\land$ et $\lor$, dont les règles se déduisent en échangeant gauche et droite.
\item La négation est une involution, i.e. $\neg\neg A \equiv A$.
\item Cette logique admet des modèles simples, les algèbres de Boole. Les formules valides en algèbre booléenne sont exactement celles démontrables en LK.
\end{itemize}

La propriété la plus importante est la \emph{propriété de la sous-formule}, vérifiée par les preuves dites \emph{sans coupure}, celles qui ne font pas appel à la règle de coupure. Cette propriété est que la démonstration ne fait pas intervenir de formules \enquote{sorties de nulle part} : les formules intermédiaires de l'arbre de preuve sont contenues dans la conclusion.

Ce qui rend d'autant plus pertinent cette propriété est le théorème d'\emph{élimination des coupures} (ou Haupsatz\footnote{\enquote{théorème fondamental} en allemand} de Gentzen) :
\[ \Gamma \vdash_{\mathrm{LK}} \Delta \Leftrightarrow 
   \Gamma \vdash_{\mathrm{LK} \setminus \mathrm{Cut}} \Delta \]
Autrement dit, tout jugement dérivable en LK admet une preuve sans coupure.

Ce théorème permet de prouver la cohérence du système, c'est-à-dire l'absence de contradiction, qu'on pourrait formuler $\not\vdash A \land \neg A$. Une preuve sans coupure de $\vdash A \land \neg A$ devrait se terminer par
\begin{prooftree}
  \AxiomC{\vdots}
  \UnaryInfC{$\vdash A$}
  \AxiomC{\vdots}
  \UnaryInfC{$A \vdash$}
  \UnaryInfC{$\vdash \neg A$}
  \BinaryInfC{$\vdash A \land \neg A$}
\end{prooftree}
or $\vdash A$ et $A \vdash$ ne peuvent être prouvés sans coupure.

\paragraph{LK est non constructive} En LK, on peut obtenir des preuves d'existence sans pouvoir obtenir un témoin de cette existence. En voici un exemple célèbre, le \enquote{drinker paradox}\footnote{le nom est dû à l'interprétation suivante en langue naturelle : dans un bar (non vide), il y a quelqu'un tel que si cette personne boit, tout le monde boit.}, qui consiste en la formule : $\exists x\, (P(x) \Rightarrow \forall y\, P(y))$.

\begin{prooftree}
    \AxiomC{}
    \UnaryInfC{$P(x), P(z) \vdash P(z), \forall y . P(y)$}
    \UnaryInfC{$P(x) \vdash P(z), P(z) \Rightarrow \forall y . P(y)$}
    \UnaryInfC{$P(x) \vdash P(z), \exists x (P(x) \Rightarrow \forall y . P(y))$}
    \UnaryInfC{$P(x) \vdash \forall y . P(y), \exists x. (P(x) \Rightarrow \forall y . P(y))$}
    \UnaryInfC{$\vdash P(x) \Rightarrow \forall y . P(y), \exists x . (P(x) \Rightarrow \forall y . P(y))$}
    \UnaryInfC{$\vdash \exists x . (P(x) \Rightarrow \forall y . P(y)), \exists x (P(y) \Rightarrow \forall y . P(y))$}
    \UnaryInfC{$\vdash \exists x . (P(x) \Rightarrow \forall y . P(y))$}
\end{prooftree}

\emph{Remarque importante :} on a employé de façon cruciale la règle d'affaiblissement ; c'est une indication du pouvoir insoupçonné des règles structurelles.

\subsubsection{Logique intuitionniste : le système LJ}

Pour obtenir une logique constructive, on peut procéder à une mutilation brutale des symétries de LK : on va interdire la présence de plusieurs formules à droite, ce qui restreint fortement la flexibilité des règles structurelles et de la manipulation de la négation. La logique obtenue ainsi est nommée LJ, c'est une formalisation de la logique \emph{intuitionniste} (elle est équivalente au système NJ de Prawitz, qui correspond au lambda-calcul typé). En LJ, un jugement est de la forme $\Gamma \vdash A$, où $\Gamma$ est un séquent et $A$ une formule, ou alors de la forme $\Gamma \vdash$.

On obtient bien la constructivité, qui s'exprime par le théorème suivant : si $\Gamma \vdash A \lor B$, alors $\Gamma \vdash A$ ou $\Gamma \vdash B$. En particulier, $\not\vdash A \lor \neg A$ ; on pourra vérifier que la preuve du tiers exclu plus haut ne marche plus.

Cependant, LJ a de nombreux inconvénients par rapport à LK. Tout d'abord, les modèles de la logique intuitionniste sont bien plus compliqués (modèles de Kripke, etc.). Surtout, on a perdu toute la symétrie : la gauche et la droite du $\vdash$ se comportent différemment, les connecteurs ne sont plus interdéfinissables, et la négation n'est plus involutive. Quel prix à payer !

Plutôt que de briser la symétrie, on peut s'y prendre en contrôlant les règles structurelles : cette alternative s'appelle la \emph{logique linéaire}.

\subsubsection{Le meilleur des deux mondes : la logique linéaire}

La logique linéaire repose sur une interprétation des formules en termes de \emph{ressources}.
\begin{itemize}
\item En logique classique, $A \Rightarrow B$ signifie simplement \enquote{si $A$, alors $B$}.
\item En logique intuitionniste, on interprète\footnote{Il s'agit de l'interprétation dite de Brower--Heyting--Kolmogorov.} $A \Rightarrow B$ comme une fonction qui prend une preuve de $A$ et renvoie une preuve de $B$. On pourrait voir ça comme une machine qui transforme une machine productrice de $A$ en machine productrice de $B$.
\item En logique linéaire, on a une notion de consommation : on va interpréter l'implication linéaire $A \multimap B$ comme \enquote{je consomme exactement 1 $A$, et je produis exactement 1 $B$}.
\end{itemize}
Selon cette interprétation \enquote{économique}, les règles de contraction et d'affaiblissement sont suspectes (on pourrait faire avec 1 $A$ la même chose qu'avec 2 $A$ ?). On va donc supprimer ces règles en logique linéaire.

En reprenant les règles logiques du système LK, on verra que la logique linéaire restaure la symétrie et la dualité. Quid du tiers exclu ? En fait, en LL, il va y avoir deux connecteurs différents correspondant au \enquote{ou} logique, notés $\parr$ et $\oplus$. Le tiers exclu dépend du \enquote{ou} : on a $\vdash A \parr \neg A$, mais $\not\vdash A \oplus \neg A$ !


\subsection{Le système LL}

Souvenons-nous des deux présentations des règles logiques en LK. En l'absence des règles structurelles, elles ne sont pas équivalentes ; on va donc se retrouver avec 2 jeux de règles, les règles multiplicatives et les règles additives, définissant 2 \enquote{et} et 2 \enquote{ou} différents.

\subsubsection{Règles logiques}

\paragraph{Axiome} S'écrit simplement\AxiomC{} \RightLabel{(Ax)} \UnaryInfC{$A \vdash A$} \DisplayProof \\
Pas de contexte supplémentaire à gauche où à droite.

\paragraph{Coupure}
\AxiomC{$\Gamma \vdash A, \Delta$}
\AxiomC{$\Gamma', A \vdash \Delta'$}
\RightLabel{(Cut)}
\BinaryInfC{$\Gamma, \Gamma' \vdash \Delta, \Delta'$}
\DisplayProof
: comme en LK

\paragraph{Connecteurs multiplicatifs}

\begin{itemize}
\item Le \enquote{et} multiplicatif est $\otimes$, prononcé \enquote{tenseur}.
\item Le \enquote{ou} multiplicatif est $\parr$,   prononcé \enquote{par}.
\end{itemize}

\begin{tabular}{ l r }

\AxiomC{$\Gamma, A, B, \vdash \Delta$}
\RightLabel{($\otimes$L)}
\UnaryInfC{$\Gamma, A \otimes B \vdash \Delta$}
\DisplayProof

&

\AxiomC{$\Gamma  \vdash A, \Delta$}
\AxiomC{$\Gamma' \vdash B, \Delta'$}
\RightLabel{($\otimes$R)}
\BinaryInfC{$\Gamma, \Gamma' \vdash A \otimes B, \Delta, \Delta'$}
\DisplayProof

\\

\AxiomC{$\Gamma,  A \vdash \Delta$}
\AxiomC{$\Gamma', B \vdash \Delta'$}
\RightLabel{($\parr$L)}
\BinaryInfC{$\Gamma, \Gamma', A \parr B \vdash \Delta, \Delta'$}
\DisplayProof

&

\AxiomC{$\Gamma  \vdash A, B, \Delta$}
\RightLabel{($\parr$R)}
\UnaryInfC{$\Gamma, \vdash A \parr B, \Delta$}
\DisplayProof

\end{tabular}


\paragraph{Connecteurs additifs}

\begin{itemize}
\item Le \enquote{et} additif est $\avec$,  prononcé \enquote{avec}.
\item Le \enquote{ou} additif est $\oplus$, prononcé (je sais pas ? somebody help me!).
\end{itemize}

\begin{tabular}{l}

\AxiomC{$\Gamma, A \vdash \Delta$}
\RightLabel{($\avec$L1)}
\UnaryInfC{$\Gamma, A \avec B \vdash \Delta$}
\DisplayProof

\,

\AxiomC{$\Gamma, B \vdash \Delta$}
\RightLabel{($\avec$L2)}
\UnaryInfC{$\Gamma, A \avec B \vdash \Delta$}
\DisplayProof

\qquad

\AxiomC{$\Gamma \vdash A, \Delta$}
\AxiomC{$\Gamma \vdash B, \Delta$}
\RightLabel{($\avec$R)}
\BinaryInfC{$\Gamma \vdash A \avec B, \Delta$}
\DisplayProof

\\

\AxiomC{$\Gamma, A \vdash \Delta$}
\AxiomC{$\Gamma, B \vdash \Delta$}
\RightLabel{($\oplus$L)}
\BinaryInfC{$\Gamma, A \oplus B \vdash \Delta$}
\DisplayProof

\qquad

\AxiomC{$\Gamma \vdash A, \Delta$}
\RightLabel{($\oplus$R1)}
\UnaryInfC{$\Gamma \vdash A \oplus B, \Delta$}
\DisplayProof

\,

\AxiomC{$\Gamma \vdash B, \Delta$}
\RightLabel{($\oplus$R2)}
\UnaryInfC{$\Gamma \vdash A \oplus B, \Delta$}
\DisplayProof

\end{tabular}

\paragraph{La négation} Elle ne change pas par rapport à LK : elle marque le passage entre les deux côtés du $\vdash$, et est involutive.

\paragraph{L'implication linéaire} Notée $\multimap$, elle est donnée par les mêmes règles que l'implication classique. Elle est multiplicative. On a $A \multimap B \equiv \neg A \parr B$.

\subsubsection{Modalités exponentielles et règles structurelles}

De la même façon qu'il est possible d'encoder la logique classique dans la logique intuitionniste par \enquote{double-negation translation}, on voudrait pouvoir plonger LJ dans LL. Pour cela, il faudrait être capable de parler de ressources qui ne coûtent rien, telles les propositions vraies de la logique intuitionniste, qui restent valables après utilisation. On va donc réintroduire la contraction et l'affaiblissement, mais de manière contrôlée, en signalant leur champ d'applicabilité par des \emph{modalités} : (note : quelle est la prononciation ?)
\begin{itemize}
\item $\ofcourse A$ : \enquote{autant de $A$ qu'on veut}
\item $\whynot A$ : \enquote{un $A$ jetable}
\end{itemize}
Bien entendu, $\ofcourse$ et $\whynot$ seront duaux. Grâce à ces modalités, on pourra exprimer l'implication intuitionniste $A \Rightarrow B = \ofcourse A \multimap B$, qui sera la traduction de l'implication de LJ.

% Noms des règles repris depuis l'article sur plato.stanford.edu

\paragraph{Contraction}
\AxiomC{$\Gamma, \ofcourse A, \ofcourse A \vdash\Delta$}
\RightLabel{($\ofcourse$C)}
\UnaryInfC{$\Gamma, \ofcourse A \vdash \Delta$}
\DisplayProof
\;
\AxiomC{$\Gamma \vdash \whynot A, \whynot A, \Delta$}
\RightLabel{($\whynot$C)}
\UnaryInfC{$\Gamma \vdash \whynot A, \Delta$}
\DisplayProof


\paragraph{Affaiblissement} 
\AxiomC{$\Gamma \vdash\Delta$}
\RightLabel{($\ofcourse$W)}
\UnaryInfC{$\Gamma, \ofcourse A \vdash \Delta$}
\DisplayProof
\;
\AxiomC{$\Gamma \vdash \Delta$}
\RightLabel{($\whynot$W)}
\UnaryInfC{$\Gamma \vdash \whynot A, \Delta$}
\DisplayProof

\paragraph{Règles d'introduction}

\begin{tabular}{ l r }

\AxiomC{$\Gamma, A \vdash \Delta$}
\RightLabel{($\ofcourse$L)}
\UnaryInfC{$\Gamma, \ofcourse A \vdash \Delta$}
\DisplayProof

&

\AxiomC{$\ofcourse \Gamma \vdash \whynot \Delta, A$}
\RightLabel{($\ofcourse$R)}
\UnaryInfC{$\ofcourse \Gamma \vdash \whynot \Delta, \ofcourse A$}
\DisplayProof

\\

\AxiomC{$\ofcourse \Gamma, A \vdash \whynot \Delta$}
\RightLabel{($\whynot$L)}
\UnaryInfC{$\ofcourse \Gamma, \whynot A \vdash \whynot \Delta$}
\DisplayProof

&

\AxiomC{$\Gamma  \vdash A, \Delta$}
\RightLabel{R)}
\UnaryInfC{$\Gamma, \vdash \whynot A, \Delta$}
\DisplayProof

\end{tabular}

(Abréviation : si $\Gamma = A_1,\ldots,A_n$, $\ofcourse \Gamma = \ofcourse A_1, \ldots, \ofcourse A_n$.)

Les règles $\ofcourse$L et $\whynot$R (qui sont duales) sont aussi appelées \emph{déréliction}.

\subsubsection{Interprétation des connecteurs}

$\parr$ : version symétrisée de l'implication (commutativité <-> contraposition).
exemple du resto. 

\subsubsection{Fragments de la logique linéaire}

MLL, MALL, MELL, LL... définition + tous admettent l'élimination des coupures.

Résultats de complexité pour la dérivabilité :
\begin{itemize}
\item dans MLL : NP-complet
\item dans MALL : PSPACE-complet
\item dans LL : indédicable
\end{itemize}

\subsubsection{Catégoricité des connecteurs}

(sans lien avec la théorie des catégories)

Un connecteur est dit \emph{catégorique} lorsqu'il est pleinement défini par ses règles d'introduction. C'est le cas des tous les connecteurs en MALL. Voici un exemple :

\paragraph{Exercice :} on définit le connecteur $\ast$ par
\AxiomC{$\Gamma,  A \vdash \Delta$}
\AxiomC{$\Gamma', B \vdash \Delta'$}
\BinaryInfC{$\Gamma, \Gamma', A \ast B \vdash \Delta, \Delta'$}
\DisplayProof
\;
\AxiomC{$\Gamma  \vdash A, B, \Delta$}
\UnaryInfC{$\Gamma, \vdash A \ast B, \Delta$}
\DisplayProof \\
Montrer que $\ast$ et $\parr$ sont équivalents.

\paragraph{} Par contre, la catégoricité n'est pas vérifiée dans certaines logiques modales, par exemple LTL\footnote{Linear Temporal Logic.}. (c'est vrai ça ? les présentations de LTL n'emploient pas des règles d'inférences pour définir les connecteurs, de toute façon...)

(exponentielles : catégoriques avec la bonne présentation ?)

Une propriété formelle qui distingue les exponentielles des connecteurs MALL, c'est la présence de contraintes imposées sur le contexte dans les règles d'inférence. Par exemple, dans l'introduction du $\ofcourse$, le contexte à gauche doit être de la forme $\ofcourse \Gamma$, et le contexte à droite doit s'écrire $\whynot \Delta$.

\subsubsection{Présentation asymétrique de LL}

En exploitant la symétrie, on va transformer les règles pour se ramener à un système où les séquents n'apparaissent qu'à droite. Par dualité, la moitié des règles est redondante, on peut ne garder que les règles droites. Puis on transforme $\Gamma \vdash \Delta$ en $\vdash \neg \Gamma, \Delta$ ; souvent, on peut purement et simplement effacer $\Gamma$.

On se retrouve alors avec un problème avec la négation : comment convertir l'introduction de la négation ? Si on passe tout à droite :
\begin{itemize}
\item soit on obtient
  \AxiomC{$\vdash \neg A, \Delta$}
  \UnaryInfC{$\vdash \neg A, \Delta$}
  \DisplayProof
  ce qui est une profonde trivialité
\item soit on obtient
  \AxiomC{$\vdash A, \Delta$}
  \UnaryInfC{$\vdash \neg \neg A, \Delta$}
  \DisplayProof
  qui ne permet d'introduire que des doubles négations : \\c'est pas terrible
\end{itemize}
Idée géniale : ne pas considérer la négation comme un connecteur, mais comme une opération !

On va prendre pour univers des atomes un ensemble infini muni d'une involution $x \mapsto x^\perp$. La négation sera alors définie par induction structurelle :
\begin{itemize}
\item Cas de base : pour un atome $x$, $x^\perp$ est l'image $x$ par l'involution.
\item Cas inductif :
  \begin{itemize}
  \item ${(F \otimes G)}^\perp = F^\perp \parr G^\perp$
  \item ${(F \oplus  G)}^\perp = F^\perp \avec G^\perp$
  \item etc.
  \end{itemize}
\end{itemize}
Notons que si on remplace ${\_}^\perp$ par $\neg \_$, on obtient des règles dérivées dans LL. Mais ici, les égalités sont des \emph{définitions}.

Ainsi, la négation devient une abréviation pour une formule où apparaissent des $\perp$ sur les atomes, elle n'existe plus comme \enquote{vrai} connecteur.

Finalement, on se retrouve avec une seule règle d'introduction par connecteur (pour $\oplus$, il y a un couple d'introductions selon que $A \oplus B$ est déduit de $A$ ou de $B$). Le système obtenu, \emph{LL asymétrique}, est plus compact que LL mais prouve exactement la même chose :
\[ \Gamma \vdash_{\mathrm{LL}} \Delta \quad \Longleftrightarrow \quad
   \vdash_{\mathrm{LL\, asym.}} \neg \Gamma, \Delta \]

La règle de coupure, qui s'écrit
\AxiomC{$\vdash A, \Delta$}
\AxiomC{$\vdash \Delta', A^\perp$}
\BinaryInfC{$\vdash \Delta, \Delta'$}
\DisplayProof,
peut bien entendu être éliminée.


\subsection{Réseaux de preuve et critères de correction}

Les réseaux de preuve sont une représentation graphique pour calculer en logique linéaire multiplicative. On représente des preuves non plus comme des arbres, mais comme des hypergraphes orientés, dont :
\begin{itemize}
\item les sommets sont des formules ;
\item les hyperarêtes sont des introductions de connecteurs.
\end{itemize}
Les formules apparaissant aux extrémités d'un réseau (c'est-à-dire les sommets de degré 1) de preuve constituent les conclusions de ce réseau.

\subsubsection{Réseaux de preuve pour MLL}

On définit inductivement les réseaux de preuve par : dessins.
On voit facilement que ces règles correspondent aux règles d'introduction en MLL asymétrique, d'où le théorème suivant : $\vdash_{\mathrm{MLL}} \Gamma$ ssi il existe un réseau de preuve avec pour conséquences $\Gamma$.

\paragraph{Qu'a-t-on gagné} Les réseaux de preuve permettent d'éliminer la \emph{bureaucratie} liée à la syntaxe.

Un exemple simple : supposons qu'on ait démontré $\vdash A, B, A', B'$. Il y a 2 façons distinctes d'en déduire $\vdash A \parr B, A' \parr B'$:
\begin{center}
  \AxiomC{$\vdash A, B, A', B'$}
  \UnaryInfC{$\vdash A \parr B, A', B'$}
  \UnaryInfC{$\vdash A \parr B, A' \parr B'$}
  \DisplayProof
  \qquad
  \AxiomC{$\vdash A, B, A', B'$}
  \UnaryInfC{$\vdash A, B, A' \parr B'$}
  \UnaryInfC{$\vdash A \parr B, A' \parr B'$}
  \DisplayProof
\end{center}
Mais moralement, ces deux preuves sont les mêmes ! La présentation en \enquote{calcul des séquents}, où on doit introduire séquentiellement chaque connecteur, nous force à faire des distinctions sans pertinence. Les deux preuves ci-dessus sont identifiées à un seul réseau de preuve.

Dessin.

\paragraph{Quelles nouvelles difficultés ?} Il faut s'assurer de la validité des réseaux de preuve. 

Avec les séquents, il était facile de vérifier la validité d'une preuve : il suffisait que les règles d'inférence soient appliquées correctement. La validité d'un réseau de preuve, quant à elle, pose problème. En effet, il existe des hypergraphes 


Proof structures.
Nécessité de critères.
Critère exponentiel : le long trip (noeuds $\to$ échangeurs).
PN pour MELL : boîtes.



\subsubsection{Premier critère : \enquote{long trip}}

Ce critère apparaît dans l'article fondateur de Girard en 1987.

Problème : trop compliqué ; exponentiel en le nombre de connecteurs.

\subsubsection{ACC et contractibilité}

\subsection{Pour aller plus loin}

\subsubsection{Géométrie de l'interaction}

\subsubsection{Réduction optimale}


\section{Les réseaux de preuve du point de vue calculatoire}

\subsection{Réduction des réseaux de preuve}

\subsubsection{Élimination des coupures}

\subsubsection{Manipulation des modalités exponentielles}

\subsection{Plongement de $\lambda$-calculs dans les réseaux de preuve : le $\lambda s$-calcul}

Théorèmes de simulation, de PSN, etc. 

\subsection{$\lambda$-calcul vers réseaux de preuve, la suite : le $\lambda x l r$-calcul}



\end{document}
